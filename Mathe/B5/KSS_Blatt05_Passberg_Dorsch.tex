 \documentclass[a4paper]{scrartcl}

\usepackage[utf8]{inputenc}
\usepackage[ngerman]{babel}

\usepackage{url,amssymb,mathrsfs,enumerate,dsfont}
\usepackage[space,extendedchars]{grffile}
\usepackage{verbatim}
\usepackage{listings}
\usepackage{geometry}
\usepackage{tikz}
\usepackage{etoolbox}
\usetikzlibrary{automata,arrows}
\usepackage{subfigure}
\usepackage[ngerman]{babel}
\usepackage{hyperref}
\usepackage{blindtext}
\usepackage{framed}
\usepackage{paralist}
\usepackage{multirow} 
\usepackage{amsmath}
\usepackage{algorithm}
\usepackage[noend]{algpseudocode}

\def\ojoin{\setbox0=\hbox{$\bowtie$}%
  \rule[-.02ex]{.25em}{.4pt}\llap{\rule[\ht0]{.25em}{.4pt}}}
\def\leftouterjoin{\mathbin{\ojoin\mkern-5.8mu\bowtie}}
\def\rightouterjoin{\mathbin{\bowtie\mkern-5.8mu\ojoin}}
\def\fullouterjoin{\mathbin{\ojoin\mkern-5.8mu\bowtie\mkern-5.8mu\ojoin}}

\usetikzlibrary{arrows,shapes, automata}
\setkomafont{disposition}{\normalfont\bfseries}
\setlength\parindent{0pt}

\lstset
{ %Formatting for code in appendix
    language=c,
    basicstyle=\footnotesize,
    numbers=left,
    stepnumber=1,
    showstringspaces=false,
    tabsize=4,
    breaklines=true,
    breakatwhitespace=false,
}

\title{Mathematik für Informatiker \\ Kombinatorik, Stochastik und Statistik}
\subtitle{Übungsblatt 5}
\author{Tom Paßberg , Iain Dorsch}
\date{}
\begin{document}

\maketitle

\newpage

\section*{Aufgabe 2}
\subsection*{a)}
\begin{lstlisting}
fn partition(n: u32, m: u32) -> Vec<Vec<u32>> {
	if m == 1 {
		return vec![vec![n]];
	}
	(1..=n-m+1).flat_map(|i| 
		partition(n-i, m-1)
			.into_iter()
			.filter(|p| p[p.len()-1] >= i)
			.map(|mut p| { p.push(i); p })
			.collect::<Vec<_>>()
	).collect()
}
\end{lstlisting}
\begin{itemize}
	\item \textbf{2..4:} Basisfall für $m = 1$.
	\item \textbf{5:} Schleife über $ i \in \{ 1,\ldots, n-m+1\} $
	\item \textbf{6:} Rekursiver Aufruf mit $n-i$ und $m-1$.
	\item \textbf{8:} Filtert alle Partitionen, bei denen das letzte Element kleiner als $i$ ist.
	\item \textbf{9:} Fügt $i$ als neues Element hinzu.
\end{itemize}
\subsection*{b)}
Ausgaben für $n = 7$ und $m \in \{1,2,3\}$:

\begin{lstlisting}
// m = 1
[7]
// m = 2
[6, 1]
[5, 2]
[4, 3]
// m = 3
[5, 1, 1]
[4, 2, 1]
[3, 3, 1]
[3, 2, 2]
\end{lstlisting}

\newpage
\section*{Aufgabe 3}
\subsection*{a)}
\begin{lstlisting}
fn partition(n: u32, m: u32) -> Vec<Vec<u32>> {
	if m == 1 {
		return vec![vec![n]];
	}
	(0..=n-m+1).flat_map(|i| 
		partition(n-i, m-1)
		.into_iter()
		.filter(|p| p[p.len()-1] >= i)
		.map(|mut p| { p.push(i); p })
		.collect::<Vec<_>>()
	).collect()
}
\end{lstlisting}
\begin{itemize}
	\item \textbf{5:} Schleife angepasst, damit auch $0$ als Element in der Partition vorkommen kann.
\end{itemize}
\subsection*{b)}
Aufgabe für $n = 7$ und $m = 3$:
\begin{lstlisting}
[7, 0, 0]
[6, 1, 0]
[5, 2, 0]
[4, 3, 0]
[5, 1, 1]	
[4, 2, 1]
[3, 3, 1]
[3, 2, 2]
\end{lstlisting}

\section*{Aufgabe 4}
\subsection*{a)}
Die Warscheinlichkeit $P(n)$, dass bei $n$ Würfen keine $6$ auftritt ist 
\begin{align*}
	P(n) = \frac{5^n}{6^n}
\end{align*}
Die Anzahl der möglichen Würfe ohne das Auftreten einer $6$ ist $5^n$, da es für jeden Wurf $5$ 
Möglichkeiten gibt, eine Zahl zwischen $1$ und $5$ zu würfeln. Die Anzahl der möglichen Würfe 
insgesamt ist $6^n$.	
\newpage
\subsection*{b)}
Die Gewinnwahrscheinlichkeit $W(n) = 1 - P(n)$ ist die Gegenwarscheinlichkeit der Wahrscheinlichkeit, 
dass bei $n$ Würfen keine $6$ auftritt. \\
Wir suchen das kleinste $n$, bei dem $W(n) > 0.5$ und damit $ P(n) < 0.5$ ist.
\begin{align*}
	0.5 &< P(n) \\
	0.5 &< \frac{5^n}{6^n} \\
	0.5 &< \left(\frac{5}{6}\right)^n \\
	\log_{\frac{5}{6}}(0.5) &< n \\
	3.80 &< n
\end{align*} 
Ab n = 4 ist die Gewinnwahrscheinlichkeit größer als $50\%$.



\end{document}