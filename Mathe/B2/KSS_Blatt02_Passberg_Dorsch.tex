\documentclass[a4paper]{scrartcl}

\usepackage[utf8]{inputenc}
\usepackage[ngerman]{babel}

\usepackage{url,amsmath,amssymb,mathrsfs,enumerate,dsfont}
\usepackage[space,extendedchars]{grffile}
\usepackage{algorithm,algorithmic}
\usepackage{verbatim}
\usepackage{listings}
\usepackage{geometry}
\usepackage{tikz}
\usepackage{etoolbox}
\usetikzlibrary{automata,arrows}
\usepackage{subfigure}
\usepackage[ngerman]{babel}
\usepackage{hyperref}
\usepackage{blindtext}
\usepackage{framed}
\usepackage{paralist}
\usepackage{multirow}

\def\ojoin{\setbox0=\hbox{$\bowtie$}%
  \rule[-.02ex]{.25em}{.4pt}\llap{\rule[\ht0]{.25em}{.4pt}}}
\def\leftouterjoin{\mathbin{\ojoin\mkern-5.8mu\bowtie}}
\def\rightouterjoin{\mathbin{\bowtie\mkern-5.8mu\ojoin}}
\def\fullouterjoin{\mathbin{\ojoin\mkern-5.8mu\bowtie\mkern-5.8mu\ojoin}}

\usetikzlibrary{arrows,shapes, automata}
\setkomafont{disposition}{\normalfont\bfseries}
\setlength\parindent{0pt}

\title{Mathematik für Informatiker \\ Kombinatorik, Stochastik und Statistik}
\subtitle{Übungsblatt 2}
\author{Tom Paßberg , Iain Dorsch}
\date{}
\begin{document}

\maketitle

\newpage
\section*{Aufgabe 1}
\subsection*{a)}

Das Problem lässt sich mit der Formel aus Aufgabe 1 b) lösen.
Hierzu setzen wir die Anzahl der Avenues auf 5 und die Anzahl der Streets auf 4.
Die kürzesten Pfade repräsenteieren die möglichen Anordungen der Warteschlangen.
Geht ein Pfad im ersten Schritt nach rechts, ist der erste Kunde ein Kunde mit einem 10€ Schein.
Geht ein Pfad im ersten Schritt nach unten, ist der erste Kunde ein Kunde mit einem 20€ Schein.
Ein Pfad ist gültig wenn er immer mindestend genauso oft nach rechts wie nach unten geht, er muss also über der Diagonale(siehe Abbildung) liegen.

\begin{center}
    \begin{tikzpicture}
        \node at (-0.5,1.5) {20};
        \node at (2,-0.5) {10};
        
        \fill[gray!50] (0,3) -- (0,0) -- (3,0) -- cycle;

        \draw[step=1cm,black,very thin] (0,0) grid (4,3);
    \end{tikzpicture}
\end{center}

Nach der Formel aus Aufgabe 1 b) ergibt sich also:
\begin{align*}
    \binom{n+m}{n} - \binom{n+m}{n+1} = \binom{7}{4} - \binom{7}{5} = 14
\end{align*}
    
\subsection*{b)}

\section*{Aufgabe 2}
\begin{enumerate}
    \item[Identität] $a \to a$
    \item[Drehung um $ 72^\circ $] $a \to \left\{ 
        \begin{array}{ c l }
            a + 1 & \quad \textrm{if } x < 5 \\
            1 & \quad \textrm{if } x = 5 \\
        \end{array}
    \right.$
    \item[Drehung um $ 144^\circ $] $a \to \left\{ 
        \begin{array}{ c l }
            a + 2 & \quad \textrm{if } x < 4 \\
            a - 3 & \quad \textrm{if } x \ge 4 \\
        \end{array}
    \right.$
    \item[Drehung um $ 216^\circ $] $a \to \left\{ 
        \begin{array}{ c l }
            a + 3 & \quad \textrm{if } x < 3 \\
            a - 2 & \quad \textrm{if } x \ge 3 \\
        \end{array}
    \right.$
    \item[Drehung um $ 288^\circ $] $a \to \left\{ 
        \begin{array}{ c l }
            a + 4 & \quad \textrm{if } x < 2 \\
            a - 1 & \quad \textrm{if } x \ge 2 \\
        \end{array}
    \right.$
    \item[Spiegelung 1] $a \to \left\{ 
        \begin{array}{ c l }
            1 & \quad \textrm{if } x = 1 \\
            5 & \quad \textrm{if } x = 2 \\
            4 & \quad \textrm{if } x = 3 \\
            3 & \quad \textrm{if } x = 4 \\
            2 & \quad \textrm{if } x = 5 \\
        \end{array}
      \right.$
    \item[Spiegelung 2] $ a \to \left\{ 
        \begin{array}{ c l }
            3 & \quad \textrm{if } x = 1 \\
            2 & \quad \textrm{if } x = 2 \\
            1 & \quad \textrm{if } x = 3 \\
            5 & \quad \textrm{if } x = 4 \\
            4 & \quad \textrm{if } x = 5 \\
        \end{array}
      \right.$ 
    \item[Spiegelung 3] $ a \to \left\{ 
        \begin{array}{ c l }
            5 & \quad \textrm{if } x = 1 \\
            4 & \quad \textrm{if } x = 2 \\
            3 & \quad \textrm{if } x = 3 \\
            2 & \quad \textrm{if } x = 4 \\
            1 & \quad \textrm{if } x = 5 \\
        \end{array}
    \right.$
    \item[Spiegelung 4] $ a \to \left\{ 
        \begin{array}{ c l }
            2 & \quad \textrm{if } x = 1 \\
            1 & \quad \textrm{if } x = 2 \\
            5 & \quad \textrm{if } x = 3 \\
            4 & \quad \textrm{if } x = 4 \\
            3 & \quad \textrm{if } x = 5 \\
        \end{array}
    \right.$
    \item[Spiegelung 5] $ a \to \left\{ 
        \begin{array}{ c l }
            4 & \quad \textrm{if } x = 1 \\
            3 & \quad \textrm{if } x = 2 \\
            2 & \quad \textrm{if } x = 3 \\
            1 & \quad \textrm{if } x = 4 \\
            5 & \quad \textrm{if } x = 5 \\
        \end{array}
    \right.$
\end{enumerate}
Die Spiegelungen $1,\ldots,5$ sind die Spiegelungen an den Geraden, die durch die Ecke und den Mittelpunkt des Fünfecks verlaufen. 

\section*{Aufgabe 3}
\subsection*{a)}
Die Warscheinlichkeit, dass kein Brief beim Richtigen Empfänger ankommt ist $37,5\%$.
Die Anzahl der Permutaionen ist $4! = 24$.
Die Anzahl der Permutationen, bei denen kein Brief beim Richtigen Empfänger ankommt ist $9$, das ergibt sich aus der Formel 
für die Anzahl der fixpunktfreien Permutationen. 
\begin{itemize}
    \item n = 1: \\ 
    $ a(n) = 0 $
    \item n = 2: \\
    $ a(n) = 1 $
    \item n = 3: \\
    $ a(n) = 2 $
    \item n = 4: \\
    $ a(n) = 9 $
\end{itemize}
Die Warscheinlichkeit ergibt sich aus
\begin{align*}
    \frac{9}{4!} = 0.375
\end{align*}

\subsection*{b)}
\begin{align*}
    a(n) = \left\{ 
        \begin{array}{ c l }
            0 & \quad \textrm{if } n = 1 \\
            n \cdot a(n-1) + (-1)^n & \quad \textrm{if } n > 1 \\
        \end{array}
    \right.
\end{align*}

\section*{Aufgabe 4}
\subsection*{a)}

\subsection*{b)}

\subsection*{c)}
\textbf{Code:}
\begin{lstlisting}
let sample_size = 100000;

let result = (6..).find(|&rounds| {
    let win_count = (0..sample_size).filter(|_| 
        (0..rounds)
            .map(|_| rand::random::<usize>() % 6)
            .unique()
            .count() == 6
        )
        .count();
    let win_rate = win_count as f64 / sample_size as f64;
    println!("Round: {}, Win rate: {}", rounds, win_rate);
    win_rate > 0.5
}).unwrap();

println!("Ab {} Runden hat der Spieler einen Vorteil. 
    Die Bank darf demnach maximal n = {} wahlen.", 
    result, result - 1);
\end{lstlisting}

\textbf{Output:}
\begin{lstlisting}
Round: 6, Win rate: 0.01505
Round: 7, Win rate: 0.05433
Round: 8, Win rate: 0.11383
Round: 9, Win rate: 0.18978
Round: 10, Win rate: 0.27061
Round: 11, Win rate: 0.3585
Round: 12, Win rate: 0.43604
Round: 13, Win rate: 0.51252
Ab 13 Runden hat der Spieler einen Vorteil. 
Die Bank darf demnach maximal n = 12 wahlen.
\end{lstlisting}

\section*{Aufgabe 5}
Schritte nach rechts werden mit 0 und Schritte nach unten mit 1 kodiert.
\subsection*{a)}
\begin{lstlisting}
fn get_paths_a(
    n: u8, m: u8, path: &mut Vec<u8>, paths: &mut Vec<Vec<u8>>
) {
    if n == 0 && m == 0 {
        paths.push(path.clone());
    } else {
        if n > 0 {
            path.push(0);
            get_paths_a(n - 1, m, path, paths);
            path.pop();
        }
        if m > 0 {
            path.push(1);
            get_paths_a(n, m - 1, path, paths);
            path.pop();
        }
    }
}
\end{lstlisting}

\subsection*{b)}
\begin{lstlisting}
fn get_paths_b(
    n: u8, m: u8, c: u8, path: &mut Vec<u8>, paths: &mut Vec<Vec<u8>>
) {
    if n == 0 && m == 0 {
        paths.push(path.clone());
    } else {
        if n > 0 {
            path.push(0);
            get_paths_b(n - 1, m, c + 1 , path, paths);
            path.pop();
        }
        if m > 0 && c > 0 {
            path.push(1);
            get_paths_b(n, m - 1, c - 1, path, paths);
            path.pop();
        }
    }
}
\end{lstlisting}    

\textbf{Funktionsaufruf:}
\begin{lstlisting}
fn main() {
    let mut all_paths = vec![];
    get_paths_a(3, 2, &mut vec![], &mut all_paths);

    let mut upper_paths = vec![];
    get_paths_b(3, 2, 0, &mut vec![], &mut upper_paths);

    println!("Aufgabenteil a: {:?}", all_paths);
    println!("Aufgabenteil b: {:?}", upper_paths);
}
\end{lstlisting}

\textbf{Output:}
\begin{lstlisting}
Aufgabenteil a: [[0, 0, 0, 1, 1], [0, 0, 1, 0, 1], [0, 0, 1, 1, 0], 
[0, 1, 0, 0, 1], [0, 1, 0, 1, 0], [0, 1, 1, 0, 0], [1, 0, 0, 0, 1], 
[1, 0, 0, 1, 0], [1, 0, 1, 0, 0], [1, 1, 0, 0, 0]]
Aufgabenteil b: [[0, 0, 0, 1, 1], [0, 0, 1, 0, 1], [0, 0, 1, 1, 0], 
[0, 1, 0, 0, 1], [0, 1, 0, 1, 0]]
\end{lstlisting}

\section*{Aufgabe 6}
\subsection*{a)}

\subsection*{b)}
Binomischer Lehrsatz:
\begin{align*}
    (a + b)^n = \sum_{k=0}^{n} \binom{n}{k} a^{n-k} b^k
\end{align*}

\textbf{Zu zeigen:}
\begin{align*}
    (a + b)^n = \sum_{k=0}^{n} \binom{n}{k} a^{n-k} b^k \implies (a + b)^n = \sum_{k=0}^{n} \binom{n}{k} a^{k} b^{n-k}
\end{align*}
\textbf{Beweis:} \\
Wir müssen zeigen, dass
\begin{align*}
    \sum_{k=0}^{n} \binom{n}{k} a^{n-k} b^k = \sum_{k=0}^{n} \binom{n}{k} a^k b^{n-k}
\end{align*}

Für $0 \le k_1 \le n$ und $k_2 = n - k_1$ gilt:
\begin{align}
    a^{n-k_1} b^{k_1} &= a^{n - k_1} b^{n - (n - k_1)} \\
    &= a^{k_2} b^{n-k_2} 
\end{align}
und 
\begin{align*}
    \binom{n}{k_1} = \binom{n}{k_2}
\end{align*}

Die beiden Summen sind also identisch, die Summanden sind in umgekehrter Reihenfolge. 
Daraus folgt, dass die zu zeigende Aussage Wahr ist.

\end{document}