\documentclass[a4paper]{scrartcl}

\usepackage[utf8]{inputenc}
\usepackage[ngerman]{babel}

\usepackage{url,amsmath,amssymb,mathrsfs,enumerate,dsfont}
\usepackage[space,extendedchars]{grffile}
\usepackage{algorithm,algorithmic}
\usepackage{verbatim}
\usepackage{listings}
\usepackage{geometry}
\usepackage{tikz}
\usepackage{etoolbox}
\usetikzlibrary{automata,arrows}
\usepackage{subfigure}
\usepackage[ngerman]{babel}
\usepackage{hyperref}
\usepackage{blindtext}
\usepackage{framed}
\usepackage{paralist}
\usepackage{multirow}

\def\ojoin{\setbox0=\hbox{$\bowtie$}%
  \rule[-.02ex]{.25em}{.4pt}\llap{\rule[\ht0]{.25em}{.4pt}}}
\def\leftouterjoin{\mathbin{\ojoin\mkern-5.8mu\bowtie}}
\def\rightouterjoin{\mathbin{\bowtie\mkern-5.8mu\ojoin}}
\def\fullouterjoin{\mathbin{\ojoin\mkern-5.8mu\bowtie\mkern-5.8mu\ojoin}}

\usetikzlibrary{arrows,shapes, automata}
\setkomafont{disposition}{\normalfont\bfseries}
\setlength\parindent{0pt}

\title{Mathematik für Informatiker \\ Kombinatorik, Stochastik und Statistik}
\subtitle{Übungsblatt 3}
\author{Tom Paßberg , Iain Dorsch}
\date{}
\begin{document}

\maketitle

\newpage
\section*{Aufgabe 1}
\subsection*{a)}
\textbf{Zu zeigen:} Für alle $ n \in \mathbb{N}_0$ gilt: 
\begin{align*}
    \sum_{j=0}^{n} \binom{n}{j} = 2^n
\end{align*}

\textbf{Beweis:} 
\begin{itemize}
    \item \textbf{Induktionsanfang:} $n = 0$
    \begin{align*}
        \sum_{j=0}^{0} \binom{0}{j} = \binom{0}{0} = 1 = 2^0
    \end{align*}
    \item \textbf{Induktionsvoraussetzung:} Für ein beliebiges, aber festes $n \in \mathbb{N}_0$ gilt:
    \begin{align*}
        \sum_{j=0}^{n} \binom{n}{j} = 2^n
    \end{align*}
    \item \textbf{Induktionsschritt:} $n \to n+1$
    \begin{align}
        \sum_{j=0}^{n+1} \binom{n+1}{j} &= 2 + \sum_{j=1}^{n} \binom{n+1}{j} \\ 
        &= 2 + \sum_{j=1}^{n} \left( \binom{n}{j-1} + \binom{n}{j} \right) \\
        &= 2 + \sum_{j=1}^{n} \binom{n}{j-1} + \sum_{j=1}^{n}\binom{n}{j} \\
        &= 2 + \sum_{j=0}^{n-1} \binom{n}{j} + \sum_{j=1}^{n} \binom{n}{j} \\
        &= \sum_{j=0}^{n} \binom{n}{j} + \sum_{j=0}^{n} \binom{n}{j} \\
        &= 2 * \sum_{j=0}^{n} \binom{n}{j} \\
        &= 2 * 2^n \\[8px]
        &= 2^{n+1}
    \end{align}
    (1) und (4) folgt aus $ \binom{n}{0} = \binom{n}{n} = 1$. \\
    (2) folgt aus Skript 1.2.14. \\
    (7) folgt aus der Induktionsvoraussetzung. 
\end{itemize}

\subsection*{b)}
\textbf{Zu zeigen:} Für alle $ n \in \mathbb{N}_0$ gilt:
\begin{align*}
    \sum_{j=0}^{n} \binom{n}{j}^2 = \binom{2n}{n}
\end{align*}
\textbf{Beweis:} 
\begin{align}
    \sum_{j=0}^{n} \binom{n}{j}^2 &= \sum_{j=0}^{n} \binom{n}{j} \binom{n}{j} \\
    &= \sum_{j=0}^{n} \binom{n}{j} \binom{n}{n-j} \\
    &= \binom{n+n}{n} \\
    &= \binom{2n}{n}
\end{align}
(10) folgt aus Skript 1.2.5. \\
(11) folgt aus Skript 1.2.12. 

\end{document}