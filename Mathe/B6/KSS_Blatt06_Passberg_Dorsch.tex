\documentclass[a4paper]{scrartcl}

\usepackage[utf8]{inputenc}
\usepackage[ngerman]{babel}

\usepackage{url,amssymb,mathrsfs,enumerate,dsfont}
\usepackage[space,extendedchars]{grffile}
\usepackage{verbatim}
\usepackage{listings}
\usepackage{geometry}
\usepackage{tikz}
\usepackage{etoolbox}
\usetikzlibrary{automata,arrows}
\usepackage{subfigure}
\usepackage[ngerman]{babel}
\usepackage{hyperref}
\usepackage{blindtext}
\usepackage{framed}
\usepackage{paralist}
\usepackage{multirow} 
\usepackage{amsmath}
\usepackage{algorithm}
\usepackage[noend]{algpseudocode}

\def\ojoin{\setbox0=\hbox{$\bowtie$}%
  \rule[-.02ex]{.25em}{.4pt}\llap{\rule[\ht0]{.25em}{.4pt}}}
\def\leftouterjoin{\mathbin{\ojoin\mkern-5.8mu\bowtie}}
\def\rightouterjoin{\mathbin{\bowtie\mkern-5.8mu\ojoin}}
\def\fullouterjoin{\mathbin{\ojoin\mkern-5.8mu\bowtie\mkern-5.8mu\ojoin}}

\usetikzlibrary{arrows,shapes, automata}
\setkomafont{disposition}{\normalfont\bfseries}
\setlength\parindent{0pt}

\lstset
{ %Formatting for code in appendix
    language=c,
    basicstyle=\footnotesize,
    numbers=left,
    stepnumber=1,
    showstringspaces=false,
    tabsize=4,
    breaklines=true,
    breakatwhitespace=false,
}

\title{Mathematik für Informatiker \\ Kombinatorik, Stochastik und Statistik}
\subtitle{Übungsblatt 6}
\author{Tom Paßberg , Iain Dorsch}
\date{}
\begin{document}

\maketitle

\newpage
\section*{Aufgabe 1}
\subsection*{a)}
Es gibt $n!$ Möglichkeiten $n$ Personen auf n Stühlen zu platzieren. Es gibt $n$ Möglichkeiten jede Anordnung zu drehen. \\
Es gibt demnach $\frac{n!}{n} = (n-1)!$ mögliche Anordnungen.

\subsection*{b)}
 
\begin{figure}[h]
\centering
\begin{tikzpicture}
\draw (0,0) -- (0,1) node[midway,left] {1} -- (1,1) node[midway,above] {2} -- (1,0) node[midway,right] {3} -- (0,0) node[midway,below] {4};
\draw (2.5,0) -- (2.5,1) node[midway,left] {1} -- (3.5,1) node[midway,above] {2} -- (3.5,0) node[midway,right] {4} -- (2.5,0) node[midway,below] {3};
\draw (5,0) -- (5,1) node[midway,left] {1} -- (6,1) node[midway,above] {3} -- (6,0) node[midway,right] {2} -- (5,0) node[midway,below] {4};
\draw (7.5,0) -- (7.5,1) node[midway,left] {1} -- (8.5,1) node[midway,above] {3} -- (8.5,0) node[midway,right] {4} -- (7.5,0) node[midway,below] {2};
\draw (10,0) -- (10,1) node[midway,left] {1} -- (11,1) node[midway,above] {4} -- (11,0) node[midway,right] {2} -- (10,0) node[midway,below] {3};
\draw (12.5,0) -- (12.5,1) node[midway,left] {1} -- (13.5,1) node[midway,above] {4} -- (13.5,0) node[midway,right] {3} -- (12.5,0) node[midway,below] {2};
\end{tikzpicture}
\caption{Mögliche Anordnungen für $\{ 1,2,3,4 \}$}
\end{figure}

\section*{Aufgabe 2}
\subsection*{a)}
Sei die Warscheinlichkeit eine 1 zu würfeln $ P(1)$. Dann gilt 
\begin{align*}
	P(2) &= 2 \cdot P(1) \\
	P(3) &= 3 \cdot P(1) \\
	P(4) &= 4 \cdot P(1) \\
	P(5) &= 5 \cdot P(1) \\
	P(6) &= 6 \cdot P(1) 
\end{align*}
Die Summe der Warscheinlichkeiten muss 1 sein, also gilt
\begin{align*}
	1 &= P(1) + 2 \cdot P(1) + 3 \cdot P(1) + 4 \cdot P(1) + 5 \cdot P(1) + 6 \cdot P(1) \\
	1 &= 21 \cdot P(1) \\
	P(1) &= \frac{1}{21}
\end{align*}
Die Warscheinlichkeiten sind
\begin{align*}
	P(1) &= \frac{1}{21} \\
	P(2) &= \frac{2}{21} \\
	P(3) &= \frac{3}{21} \\
	P(4) &= \frac{4}{21} \\
	P(5) &= \frac{5}{21} \\
	P(6) &= \frac{6}{21}
\end{align*}

\subsection*{b)}
Die Warscheinlichkeit eine ungerade Zahl zu würfeln ist 
\begin{align*}
	P(1) + P(3) + P(5) &= \frac{1}{21} + \frac{3}{21} + \frac{5}{21} \\
	&= \frac{3}{7}
\end{align*}

\section*{Aufgabe 3}
\subsection*{a)}
\textbf{Induktionsanfang} \\
Für $m = 1$ ist die Warscheinlichkeit mit einem $n$-seitigen Würfel eine 1 zu würfeln $\frac{1}{n}$. \\
Die Warscheinlichkeit keine 1 zu würfeln ist $P(1) = 1 - \frac{1}{n} = \left(1 - \frac{1}{n}\right)^1$ \\
\textbf{Induktionsvoraussetzung} \\
Für ein beliebiges aber festes $m$ ist die Warscheinlichkeit mit einem $n$-seitigen Würfel mit $m$ 
würfen keine 1 zu würfeln
\begin{center}
	$P(m) = \left(1 - \frac{1}{n}\right)^m$ 
\end{center}
\textbf{Induktionsschritt $m \to m+1$} \\
Jeder weitere Wurf hat eine Warscheinlichkeit von $1 - \frac{1}{n}$ keine 1 zu würfeln. 
\begin{align}
	P(m+1) &= P(m) \cdot \left(1 - \frac{1}{n}\right) \\
	&= \left(1 - \frac{1}{n}\right)^m \cdot \left(1 - \frac{1}{n}\right) \\
	&= \left(1 - \frac{1}{n}\right)^{m+1}
\end{align}
(2) gilt nach der Induktionsvoraussetzung.

\subsection*{b)}
\textbf{Zu zeigen:} 
\begin{align*}
\lim_{n \to \infty} \left(1 - \frac{1}{n}\right)^{n \cdot \ln(2)} = \frac{1}{2}
\end{align*}

\textbf{Beweis:} 
\begin{align}
\lim_{n \to \infty} \left(1 - \frac{1}{n}\right)^{n \cdot \ln(2)} &= \lim_{n \to \infty} e^{n \cdot \ln(2) \cdot  \ln(1-\frac{1}{n})} \\
&= \lim_{n \to \infty} 2^{n \cdot \ln(1-\frac{1}{n})} \\
&= \lim_{n \to \infty} 2^{\ln((1-\frac{1}{n})^n)} \\
&= 2^{\ln(\lim_{n \to \infty} (1+\frac{-1}{n})^n)} \\
&= 2^{\ln(e^{-1})} \\
&= 2^{-1} \\
&= \frac{1}{2}
\end{align}
(4) folgt aus dem Hinweis auf dem Übungsblatt. \\ 
(8) folgt aus Skript Lemma 3.8.4. 

\section*{Aufgabe 4}
\textbf{Programmcode in Rust:}
\begin{lstlisting}
fn multimenge(input: &Vec<u8>, n: u8) -> Vec<Vec<u8>> {
	if n == 0 {
		return vec![vec![]];
	}
	if n == 1 {
		return input.iter().map(|&x| vec![x]).collect();
	}
	multimenge(input, n - 1).iter().flat_map(|x| 
		input.iter()
			.filter(|&y| x[x.len() - 1] <= *y)
			.map(|y: &u8| {
				let mut z = x.clone();
				z.push(*y);
				z
			})
	).collect()
}
\end{lstlisting}
\newpage
\textbf{Funktionsaufruf:}
\begin{lstlisting}
fn main() {
	let result = multimenge(&vec![1,2,3], 3);

	for i in result {
		println!("{:?}", i);
	}
}
\end{lstlisting}
\textbf{Ausgabe:}
\begin{lstlisting}
[1, 1, 1]
[1, 1, 2]
[1, 1, 3]
[1, 2, 2]
[1, 2, 3]
[1, 3, 3]
[2, 2, 2]
[2, 2, 3]
[2, 3, 3]
[3, 3, 3]
\end{lstlisting}

\end{document}