\documentclass[a4paper]{scrartcl}

\usepackage[utf8]{inputenc}
\usepackage[ngerman]{babel}

\usepackage{url,amsmath,amssymb,mathrsfs,enumerate,dsfont}
\usepackage[space,extendedchars]{grffile}
\usepackage{algorithm,algorithmic}
\usepackage{verbatim}
\usepackage{listings}
\usepackage{geometry}
\usepackage{tikz}
\usepackage{etoolbox}
\usetikzlibrary{automata,arrows}
\usepackage{subfigure}
\usepackage[ngerman]{babel}
\usepackage{hyperref}
\usepackage{blindtext}
\usepackage{framed}
\usepackage{paralist}
\usepackage{multirow}

\def\ojoin{\setbox0=\hbox{$\bowtie$}%
  \rule[-.02ex]{.25em}{.4pt}\llap{\rule[\ht0]{.25em}{.4pt}}}
\def\leftouterjoin{\mathbin{\ojoin\mkern-5.8mu\bowtie}}
\def\rightouterjoin{\mathbin{\bowtie\mkern-5.8mu\ojoin}}
\def\fullouterjoin{\mathbin{\ojoin\mkern-5.8mu\bowtie\mkern-5.8mu\ojoin}}

\usetikzlibrary{arrows,shapes, automata}
\setkomafont{disposition}{\normalfont\bfseries}
\setlength\parindent{0pt}

\title{Mathematik für Informatiker \\ Kombinatorik, Stochastik und Statistik \\ Ubungsblatt 1}
\author{Tom Paßberg , Iain Dorsch}
\date{}
\begin{document}

\maketitle

\newpage
\section*{Aufgabe 5}
Funktion um die Anzahl der Zahlen zwischen 1 und n zu berechnen, die durch mindestens einen der Teiler teilbar sind.
\begin{lstlisting}
use rayon::iter::{IntoParallelIterator, ParallelIterator};

fn count_numbers(n: u64, teiler: &Vec<u64>) -> usize {
    (1..=n).into_par_iter()
        .filter(|&n| teiler.iter().any(|&t| n % t == 0))
        .count()
}   
\end{lstlisting}

Output für $ n = 10, 100 , \ldots, 10000000000 $:

\begin{lstlisting}
fn main() {
    let teiler: Vec<u64> = vec![3,5,7,11];
    for n in (1..=10).map(|i| 10u64.pow(i)) {
        println!(
            "{:11} gerade Zahlen zwischen 1 und {n:11} 
            sind durch mindestens einen der Teiler {} teilbar.", 
            count_numbers(n, &teiler), 
            teiler.iter().map(|&t| 
                t.to_string()).collect::<Vec<String>>().join(", ")
        );
    }  
} 
\end{lstlisting}

Output:

\begin{lstlisting}
6 Zahlen zwischen 1 und 10 sind durch mindestens einen
der Teiler 3, 5, 7, 11 teilbar.
59 Zahlen zwischen 1 und 100 sind durch mindestens einen
der Teiler 3, 5, 7, 11 teilbar.
585 Zahlen zwischen 1 und 1000 sind durch mindestens einen
der Teiler 3, 5, 7, 11 teilbar.
5845 Zahlen zwischen 1 und 10000 sind durch mindestens einen
der Teiler 3, 5, 7, 11 teilbar.
58441 Zahlen zwischen 1 und 100000 sind durch mindestens einen
der Teiler 3, 5, 7, 11 teilbar.
584416 Zahlen zwischen 1 und 1000000 sind durch mindestens einen
der Teiler 3, 5, 7, 11 teilbar.
5844156 Zahlen zwischen 1 und 10000000 sind durch mindestens einen
der Teiler 3, 5, 7, 11 teilbar.
58441559 Zahlen zwischen 1 und 100000000 sind durch mindestens einen
der Teiler 3, 5, 7, 11 teilbar.
584415585 Zahlen zwischen 1 und 1000000000 sind durch mindestens einen
der Teiler 3, 5, 7, 11 teilbar.
5844155845 Zahlen zwischen 1 und 10000000000 sind durch mindestens einen
der Teiler 3, 5, 7, 11 teilbar.
\end{lstlisting}      

\end{document}
