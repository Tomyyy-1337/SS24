\documentclass[a4paper]{scrartcl}

\usepackage[utf8]{inputenc}
\usepackage[ngerman]{babel}

\usepackage{url,amsmath,amssymb,mathrsfs,enumerate,dsfont}
\usepackage[space,extendedchars]{grffile}
\usepackage{algorithm,algorithmic}
\usepackage{verbatim}
\usepackage{listings}
\usepackage{geometry}
\usepackage{tikz}
\usepackage{etoolbox}
\usetikzlibrary{automata,arrows}
\usepackage{subfigure}
\usepackage[ngerman]{babel}
\usepackage{hyperref}
\usepackage{blindtext}
\usepackage{framed}
\usepackage{paralist}
\usepackage{multirow}

\def\ojoin{\setbox0=\hbox{$\bowtie$}%
  \rule[-.02ex]{.25em}{.4pt}\llap{\rule[\ht0]{.25em}{.4pt}}}
\def\leftouterjoin{\mathbin{\ojoin\mkern-5.8mu\bowtie}}
\def\rightouterjoin{\mathbin{\bowtie\mkern-5.8mu\ojoin}}
\def\fullouterjoin{\mathbin{\ojoin\mkern-5.8mu\bowtie\mkern-5.8mu\ojoin}}

\usetikzlibrary{arrows,shapes, automata}
\setkomafont{disposition}{\normalfont\bfseries}
\setlength\parindent{0pt}

\title{Mathematik für Informatiker \\ Kombinatorik, Stochastik und Statistik \\ Ubungsblatt 1}
\author{Tom Paßberg , Iain Dorsch}
\date{}
\begin{document}

\maketitle

\newpage

\section*{Aufgabe 3}
\begin{align*}
    a &= |\{ x \mod 3 = 0 \mid 0 \leq x \leq 100000 \}| = \frac{100000}{3} \\
    b &= |\{ x \mod 5 = 0 \mid 0 \leq x \leq 100000 \}| = \frac{100000}{5} \\
    c &= |\{ x \mod 7 = 0 \mid 0 \leq x \leq 100000 \}| = \frac{100000}{7} \\
    d &= |\{ x \mod 11 = 0 \mid 0 \leq x \leq 100000 \}| = \frac{100000}{11} \\
    ab &= |\{ x \mod 3 = 0 \land x \mod 5 = 0 \mid 0 \leq x \leq 100000 \}| = \frac{100000}{3 \cdot 5} \\
    ac &= |\{ x \mod 3 = 0 \land x \mod 7 = 0 \mid 0 \leq x \leq 100000 \}| = \frac{100000}{3 \cdot 7} \\
    ad &= |\{ x \mod 3 = 0 \land x \mod 11 = 0 \mid 0 \leq x \leq 100000 \}| = \frac{100000}{3 \cdot 11} \\
    bc &= |\{ x \mod 5 = 0 \land x \mod 7 = 0 \mid 0 \leq x \leq 100000 \}| = \frac{100000}{5 \cdot 7} \\
    bd &= |\{ x \mod 5 = 0 \land x \mod 11 = 0 \mid 0 \leq x \leq 100000 \}| = \frac{100000}{5 \cdot 11} \\
    cd &= |\{ x \mod 7 = 0 \land x \mod 11 = 0 \mid 0 \leq x \leq 100000 \}| = \frac{100000}{7 \cdot 11} \\
    abc &= |\{ x \mod 3 = 0 \land x \mod 5 = 0 \land x \mod 7 = 0 \mid 0 \leq x \leq 100000 \}| = \frac{100000}{3 \cdot 5 \cdot 7} \\
    abd &= |\{ x \mod 3 = 0 \land x \mod 5 = 0 \land x \mod 11 = 0 \mid 0 \leq x \leq 100000 \}| = \frac{100000}{3 \cdot 5 \cdot 11} \\
    acd &= |\{ x \mod 3 = 0 \land x \mod 7 = 0 \land x \mod 11 = 0 \mid 0 \leq x \leq 100000 \}| = \frac{100000}{3 \cdot 7 \cdot 11} \\
    bcd &= |\{ x \mod 5 = 0 \land x \mod 7 = 0 \land x \mod 11 = 0 \mid 0 \leq x \leq 100000 \}| = \frac{100000}{5 \cdot 7 \cdot 11} \\
    abcd &= |\{ x \mod 3 = 0 \land x \mod 5 = 0 \land x \mod 7 = 0 \land x \mod 11 = 0 \mid 0 \leq x \leq 100000 \}| \\&= \frac{100000}{3 \cdot 5 \cdot 7 \cdot 11} \\
    \text{result} &= a + b + c + d - ab - ac - ad - bc - bd - cd + abc + abd + acd + bcd - abcd \\&= 58441 \\
\end{align*}

\newpage
\section*{Aufgabe 4}
\subsection*{a)}
Für jede weiter Stufe wird die voerherige Mengen kopiert und um das nächste Element erweitert.\\
Die neuen und die alten Elemente werden in einer neuen Menge zusammengefasst.\\
\begin{itemize}
    \item $n = 0$ \\
    $ \{ \{\} \} $
    \item $n = 1$ \\
    $ \{ \{\}, \{1\} \} $
    \item $n = 2$ \\
    $ \{ \{\}, \{1\}, \{2\}, \{1, 2\} \} $
    \item $n = 3$ \\
    $ \{ \{\}, \{1\}, \{2\}, \{1, 2\}, \{3\}, \{1, 3\}, \{2, 3\}, \{1, 2, 3\} \} $
    \item $n = 4$ \\
    $ \{ \{\}, \{1\}, \{2\}, \{1, 2\}, \{3\}, \{1, 3\}, \{2, 3\}, \{1, 2, 3\}, \\ \{4\}, \{1, 4\}, \{2, 4\}, \{1, 2, 4\}, \{3, 4\}, \{1, 3, 4\}, \{2, 3, 4\}, \{1, 2, 3, 4\} \} $
\end{itemize}

\subsection*{b)}
\subsection*{Beschreibung}
Für $n = 0$ gibt der Algorithmus eine Liste mit einer leeren Liste zurück. (Eine Menge in der sich eine leere Menge befindet)\\
Für $n > 0$ wird die Funktion rekursiv aufgerufen um die Teilmengen von $\{1, \ldots, n-1\}$ zu berechnen.\\
Die Teilmengen von $\{1, \ldots, n\}$ ergeben sich aus den Teilmengen von $\{1, \ldots, n-1\}$, indem von jeder Teilmengen aus $\{1, \ldots, n-1\}$ eine Kopie erstellt wird, in die n hinzugefügt wird.\\
\newpage
\subsection*{Funktion zur Berechnung der Teilmengen von $\{1, \ldots, n\}$}
\begin{lstlisting}
fn teilmengen_rec(n: u8) -> Vec<Vec<u8>> {
    if n == 0 {
        return vec![vec![]];
    } 
    teilmengen_rec(n - 1)
        .into_iter()
        .flat_map(|old| {
            let mut new = old.clone();
            new.push(n);
            [old, new]
        })
        .collect()
}
\end{lstlisting}

\subsection*{Funktiosnaufrufe für $n = 1, \ldots, 20$}
\begin{lstlisting}
fn main() {
    for i in 1..=20 {
        let teilmengen = teilmengen_rec(i);
        println!(
            "Anzahl der Teilmengen von {{1,..,{}}}: {}", 
            i, teilmengen.len()
        );
        if i <= 4 {
            println!("Teilmengen von {{1,..,{}}}: {:?}", i, teilmengen);
        }
    }  
}
\end{lstlisting}

\newpage
\subsection*{Output:}
\begin{lstlisting}
Anzahl der Teilmengen von {1,..,1}: 2
Teilmengen von {1,..,1}: [[], [1]]
Anzahl der Teilmengen von {1,..,2}: 4
Teilmengen von {1,..,2}: [[], [2], [1], [1, 2]]
Anzahl der Teilmengen von {1,..,3}: 8
Teilmengen von {1,..,3}: [[], [3], [2], [2, 3], [1], 
[1, 3], [1, 2], [1, 2, 3]]
Anzahl der Teilmengen von {1,..,4}: 16
Teilmengen von {1,..,4}: [[], [4], [3], [3, 4], [2], 
[2, 4], [2, 3], [2, 3, 4], [1], [1, 4], [1, 3], [1, 3, 4], 
[1, 2], [1, 2, 4], [1, 2, 3], [1, 2, 3, 4]]
Anzahl der Teilmengen von {1,..,5}: 32
Anzahl der Teilmengen von {1,..,6}: 64
Anzahl der Teilmengen von {1,..,7}: 128
Anzahl der Teilmengen von {1,..,8}: 256
Anzahl der Teilmengen von {1,..,9}: 512
Anzahl der Teilmengen von {1,..,10}: 1024
Anzahl der Teilmengen von {1,..,11}: 2048
Anzahl der Teilmengen von {1,..,12}: 4096
Anzahl der Teilmengen von {1,..,13}: 8192
Anzahl der Teilmengen von {1,..,14}: 16384
Anzahl der Teilmengen von {1,..,15}: 32768
Anzahl der Teilmengen von {1,..,16}: 65536
Anzahl der Teilmengen von {1,..,17}: 131072
Anzahl der Teilmengen von {1,..,18}: 262144
Anzahl der Teilmengen von {1,..,19}: 524288
Anzahl der Teilmengen von {1,..,20}: 1048576
\end{lstlisting}

\newpage
\section*{Aufgabe 5}
Funktion um die Anzahl der Zahlen zwischen 1 und n zu berechnen, die durch mindestens einen der Teiler teilbar sind.
\begin{lstlisting}
use rayon::iter::{IntoParallelIterator, ParallelIterator};

fn count_numbers(n: u64, teiler: &Vec<u64>) -> usize {
    (1..=n).into_par_iter()
        .filter(|&n| teiler.iter().any(|&t| n % t == 0))
        .count()
}   
\end{lstlisting}

Funktiosnaufrufe für $ n = 10, 100 , \ldots, 10000000000 $ auf:

\begin{lstlisting}
fn main() {
    let teiler: Vec<u64> = vec![3,5,7,11];
    for n in (1..=10).map(|i| 10u64.pow(i)) {
        println!(
            "{:11} gerade Zahlen zwischen 1 und {n:11} 
            sind durch mindestens einen der Teiler {} teilbar.", 
            count_numbers(n, &teiler), 
            teiler.iter().map(|&t| 
                t.to_string()).collect::<Vec<String>>().join(", ")
        );
    }  
} 
\end{lstlisting}

Output:

\begin{lstlisting}
6 Zahlen zwischen 1 und 10 sind durch mindestens einen
der Teiler 3, 5, 7, 11 teilbar.
59 Zahlen zwischen 1 und 100 sind durch mindestens einen
der Teiler 3, 5, 7, 11 teilbar.
585 Zahlen zwischen 1 und 1000 sind durch mindestens einen
der Teiler 3, 5, 7, 11 teilbar.
5845 Zahlen zwischen 1 und 10000 sind durch mindestens einen
der Teiler 3, 5, 7, 11 teilbar.
58441 Zahlen zwischen 1 und 100000 sind durch mindestens einen
der Teiler 3, 5, 7, 11 teilbar.
584416 Zahlen zwischen 1 und 1000000 sind durch mindestens einen
der Teiler 3, 5, 7, 11 teilbar.
5844156 Zahlen zwischen 1 und 10000000 sind durch mindestens einen
der Teiler 3, 5, 7, 11 teilbar.
58441559 Zahlen zwischen 1 und 100000000 sind durch mindestens einen
der Teiler 3, 5, 7, 11 teilbar.
584415585 Zahlen zwischen 1 und 1000000000 sind durch mindestens einen
der Teiler 3, 5, 7, 11 teilbar.
5844155845 Zahlen zwischen 1 und 10000000000 sind durch mindestens einen
der Teiler 3, 5, 7, 11 teilbar.
\end{lstlisting}      

\end{document}
