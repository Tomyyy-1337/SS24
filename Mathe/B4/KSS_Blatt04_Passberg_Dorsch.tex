 \documentclass[a4paper]{scrartcl}

\usepackage[utf8]{inputenc}
\usepackage[ngerman]{babel}

\usepackage{url,amssymb,mathrsfs,enumerate,dsfont}
\usepackage[space,extendedchars]{grffile}
\usepackage{verbatim}
\usepackage{listings}
\usepackage{geometry}
\usepackage{tikz}
\usepackage{etoolbox}
\usetikzlibrary{automata,arrows}
\usepackage{subfigure}
\usepackage[ngerman]{babel}
\usepackage{hyperref}
\usepackage{blindtext}
\usepackage{framed}
\usepackage{paralist}
\usepackage{multirow} 
\usepackage{amsmath}
\usepackage{algorithm}
\usepackage[noend]{algpseudocode}

\def\ojoin{\setbox0=\hbox{$\bowtie$}%
  \rule[-.02ex]{.25em}{.4pt}\llap{\rule[\ht0]{.25em}{.4pt}}}
\def\leftouterjoin{\mathbin{\ojoin\mkern-5.8mu\bowtie}}
\def\rightouterjoin{\mathbin{\bowtie\mkern-5.8mu\ojoin}}
\def\fullouterjoin{\mathbin{\ojoin\mkern-5.8mu\bowtie\mkern-5.8mu\ojoin}}

\usetikzlibrary{arrows,shapes, automata}
\setkomafont{disposition}{\normalfont\bfseries}
\setlength\parindent{0pt}

\title{Mathematik für Informatiker \\ Kombinatorik, Stochastik und Statistik}
\subtitle{Übungsblatt 3}
\author{Tom Paßberg , Iain Dorsch}
\date{}
\begin{document}

\maketitle

\newpage
\section*{Aufgabe 1}
\subsection*{a)}
\textbf{Zu zeigen:} Für alle $ n \in \mathbb{N}_0$ gilt: 
\begin{align*}
    \sum_{j=0}^{n} \binom{n}{j} = 2^n
\end{align*}

\textbf{Beweis:} 
\begin{itemize}
    \item \textbf{Induktionsanfang:} $n = 0$
    \begin{align*}
        \sum_{j=0}^{0} \binom{0}{j} = \binom{0}{0} = 1 = 2^0
    \end{align*}
    \item \textbf{Induktionsvoraussetzung:} Für ein beliebiges, aber festes $n \in \mathbb{N}_0$ gilt:
    \begin{align*}
        \sum_{j=0}^{n} \binom{n}{j} = 2^n
    \end{align*}
    \item \textbf{Induktionsschritt:} $n \to n+1$
    \begin{align}
        \sum_{j=0}^{n+1} \binom{n+1}{j} &= 2 + \sum_{j=1}^{n} \binom{n+1}{j} \\ 
        &= 2 + \sum_{j=1}^{n} \left( \binom{n}{j-1} + \binom{n}{j} \right) \\
        &= 2 + \sum_{j=1}^{n} \binom{n}{j-1} + \sum_{j=1}^{n}\binom{n}{j} \\
        &= 2 + \sum_{j=0}^{n-1} \binom{n}{j} + \sum_{j=1}^{n} \binom{n}{j} \\
        &= \sum_{j=0}^{n} \binom{n}{j} + \sum_{j=0}^{n} \binom{n}{j} \\
        &= 2 \cdot \sum_{j=0}^{n} \binom{n}{j} \\
        &= 2 \cdot 2^n \\[8px]
        &= 2^{n+1}
    \end{align}
    (1) und (5) folgt aus $ \binom{n}{0} = \binom{n}{n} = 1$. \\
    (2) folgt aus Skript 1.2.14. \\
    (7) folgt aus der Induktionsvoraussetzung. 
\end{itemize}

\subsection*{b)}
\textbf{Zu zeigen:} Für alle $ n \in \mathbb{N}_0$ gilt:
\begin{align*}
    \sum_{j=0}^{n} \binom{n}{j}^2 = \binom{2n}{n}
\end{align*}
\textbf{Beweis:} 
\begin{align}
    \sum_{j=0}^{n} \binom{n}{j}^2 &= \sum_{j=0}^{n} \binom{n}{j} \binom{n}{j} \\
    &= \sum_{j=0}^{n} \binom{n}{j} \binom{n}{n-j} \\
    &= \binom{n+n}{n} \\
    &= \binom{2n}{n}
\end{align}
(10) folgt aus Skript 1.2.5. \\
(11) folgt aus Skript 1.2.12. 

\section*{Aufgabe 2}

\subsection*{a)}
\begin{algorithm}
\caption{Find Partitions}\label{euclid}
\begin{algorithmic}[1]
    \Procedure{Partitions}{n,m}
    \If {$ m = 1 $} 
        \Return [[n]] 
    \EndIf
    \State $result \gets$ []
    \For {$i \gets 1$ \textbf{to} $n-m+1$}
        \For {$p \in \textsc{Partitions}(n-i,m-1)$}
            \If {$p[p.\text{len}()-1] \geq i$}
                \State $p.\text{push}(i)$
                \State $result.\text{push}(p)$
            \EndIf
        \EndFor
    \EndFor
    \State \Return $result$
    \EndProcedure
\end{algorithmic}
\end{algorithm}
\newpage
\subsection*{b)}
\begin{itemize}
    \item $[5, 1, 1]$
    \begin{center}    
        \begin{tikzpicture}
            \draw (0,3) -- (1,3) -- (1,2) -- (0,2) -- (0,3);
            \draw (1,3) -- (2,3) -- (2,2) -- (1,2) -- (1,3);
            \draw (2,3) -- (3,3) -- (3,2) -- (2,2) -- (2,3);
            \draw (3,3) -- (4,3) -- (4,2) -- (3,2) -- (3,3);
            \draw (4,3) -- (5,3) -- (5,2) -- (4,2) -- (4,3);
            \draw (0,2) -- (1,2) -- (1,1) -- (0,1) -- (0,2);
            \draw (0,1) -- (1,1) -- (1,0) -- (0,0) -- (0,1);
        \end{tikzpicture}
    \end{center}
    \item $[4, 2, 1]$
    \begin{center}    
        \begin{tikzpicture}
            \draw (0,3) -- (1,3) -- (1,2) -- (0,2) -- (0,3);
            \draw (1,3) -- (2,3) -- (2,2) -- (1,2) -- (1,3);
            \draw (2,3) -- (3,3) -- (3,2) -- (2,2) -- (2,3);
            \draw (3,3) -- (4,3) -- (4,2) -- (3,2) -- (3,3);
            \draw (0,2) -- (1,2) -- (1,1) -- (0,1) -- (0,2);
            \draw (1,2) -- (2,2) -- (2,1) -- (1,1) -- (1,2);
            \draw (0,1) -- (1,1) -- (1,0) -- (0,0) -- (0,1);
        \end{tikzpicture}
    \end{center}
    \item $[3, 3, 1]$
    \begin{center}    
        \begin{tikzpicture}
            \draw (0,3) -- (1,3) -- (1,2) -- (0,2) -- (0,3);
            \draw (1,3) -- (2,3) -- (2,2) -- (1,2) -- (1,3);
            \draw (2,3) -- (3,3) -- (3,2) -- (2,2) -- (2,3);
            \draw (0,2) -- (1,2) -- (1,1) -- (0,1) -- (0,2);
            \draw (1,2) -- (2,2) -- (2,1) -- (1,1) -- (1,2);
            \draw (2,2) -- (3,2) -- (3,1) -- (2,1) -- (2,2);
            \draw (0,1) -- (1,1) -- (1,0) -- (0,0) -- (0,1);
        \end{tikzpicture}
    \end{center}
    \item $[3, 2, 2]$
    \begin{center}    
        \begin{tikzpicture}
            \draw (0,3) -- (1,3) -- (1,2) -- (0,2) -- (0,3);
            \draw (1,3) -- (2,3) -- (2,2) -- (1,2) -- (1,3);
            \draw (2,3) -- (3,3) -- (3,2) -- (2,2) -- (2,3);
            \draw (0,2) -- (1,2) -- (1,1) -- (0,1) -- (0,2);
            \draw (1,2) -- (2,2) -- (2,1) -- (1,1) -- (1,2);
            \draw (0,1) -- (1,1) -- (1,0) -- (0,0) -- (0,1);
            \draw (1,1) -- (2,1) -- (2,0) -- (1,0) -- (1,1);
        \end{tikzpicture}
    \end{center}
\end{itemize}

\newpage
\section*{Aufgabe 3}
\textbf{Zu zeigen:} Für alle $ n, m \in \mathbb{N}$ gilt für die Anzahl der geordneten Zahlpartitionen von $n$ in $m$ positive Summanden:
\begin{align*}
    P(n,m) = \binom{n-1}{m-1}
\end{align*}
\textbf{Beweis:}
\begin{itemize}
    \item Induktionsanfang: $m = 1$: \\
    Für $ m = 1 $ ist die Anzahl der möglichen Partitionen von $ n $ in $ m $ positive Summanden gleich 1. \\
    \begin{align*}
        P(n,1) = 1 = \binom{n-1}{0}
    \end{align*}
    \item Induktionsvoraussetzung: \\
    Die Aussage gilt für ein beliebige aber feste $ n, m \in \mathbb{N}$ und für alle $ n', m' \in \mathbb{N} $ mit $ n' < n $ und $ m' < m $.
    \item Induktionsschritt:  \\
    Die Anzahl der geordneten Zahlpartitionen von $ n $ in $ m $ positive Summanden lässt sich rekursiv berechnen aus der Summe der Anzahlen der geordneten 
    Zahlpartitionen von $ n-i $ in $ m-1 $ positive Summanden für $ i = 1, \dots, n-m+1 $. Daraus folgt:
    \begin{align}
        P(n,m) &= \sum_{i=1}^{n-m+1} P(n-i, m-1) \\
        &= \sum_{i=1}^{n-m+1} \binom{n-i-1}{m-2} \\
        &= \sum_{i=0}^{n-m} \binom{m-2+i}{m-2} \\
        &= \binom{m-2+n-m+1}{m-2+1} \\
        &= \binom{n-1}{m-1}
    \end{align}
    (14) folgt aus der Induktionsvoraussetzung. \\
    (15) invertiert die Reihenfolge der Summanden \\
    (16) folgt aus $ \sum_{k=0}^m \binom{n + k}{n} = \binom{n + m + 1}{n + 1} $
\end{itemize}
    

\section*{Aufgabe 4}
\subsection*{a)}
Äquivalenzrelationen auf $M = \{1,2,3,4\}$:
\begin{align*}
    R_1 &= \{(a,a) \mid a \in M \} \\
    R_2 &= \{(a,b) \in M^2 \} \\
    R_3 &= \{(a,b) \mid a,b \in \{ 1 \} \lor a, b \in \{2, 3, 4\}  \} \\
    R_4 &= \{(a,b) \mid a,b \in \{ 2 \} \lor a, b \in \{1, 3, 4\}  \} \\
    R_5 &= \{(a,b) \mid a,b \in \{ 3 \} \lor a, b \in \{1, 2, 4\}  \} \\
    R_6 &= \{(a,b) \mid a,b \in \{ 4 \} \lor a, b \in \{1, 2, 3\}  \} \\
    R_7 &= \{(a,b) \mid a,b \in \{ 1,2 \} \lor a, b \in \{3 ,4\}  \} \\
    R_8 &= \{(a,b) \mid a,b \in \{ 1,3 \} \lor a, b \in \{2 ,4\}  \} \\
    R_9 &= \{(a,b) \mid a,b \in \{ 1,4 \} \lor a, b \in \{2 ,3\}  \} \\
    R_{10} &= \{(a,b) \mid a,b \in \{ 1 \} \lor a, b \in \{2\} \lor a,b \in \{ 3,4 \} \} \\
    R_{11} &= \{(a,b) \mid a,b \in \{ 1 \} \lor a, b \in \{3\} \lor a,b \in \{ 2,4 \} \} \\
    R_{12} &= \{(a,b) \mid a,b \in \{ 1 \} \lor a, b \in \{4\} \lor a,b \in \{ 2,3 \} \} \\
    R_{13} &= \{(a,b) \mid a,b \in \{ 2 \} \lor a, b \in \{3\} \lor a,b \in \{ 1,4 \} \} \\
    R_{14} &= \{(a,b) \mid a,b \in \{ 2 \} \lor a, b \in \{4\} \lor a,b \in \{ 1,3 \} \} \\
    R_{15} &= \{(a,b) \mid a,b \in \{ 3 \} \lor a, b \in \{4\} \lor a,b \in \{ 1,2 \} \} \\
\end{align*}

\subsection*{b)}

\subsection*{c)}
\begin{itemize}
    \item $B_0 = 1$
    \item $B_1 = 1$
    \item $B_2 = 2$
    \item $B_3 = 5$
    \item $B_4 = 15$
\end{itemize}

\newpage
\section*{Aufgabe 5}
\textbf{Code:}
\begin{lstlisting}
fn partition(n: u32, m: u32) -> Vec<Vec<u32>> {
    if m == 1 {
        return vec![vec![n]];
    }
    (1..=n-m+1).flat_map(|i| 
        partition(n-i, m-1)
            .into_iter()
            .filter(|p| p[p.len()-1] >= i)
            .map(|mut p| { p.push(i); p })
            .collect::<Vec<_>>()
    ).collect()
}
\end{lstlisting}
\textbf{Funktionsaufruf:}
\begin{lstlisting}
fn main() {
    let n = 8;
    let m = 4;
    let result = partition(n, m);
    for p in result {
        println!("{:?}", p);
    }
}
\end{lstlisting}

\textbf{Ausgabe:}
\begin{lstlisting}
[5, 1, 1, 1]
[4, 2, 1, 1]
[3, 3, 1, 1]
[3, 2, 2, 1]
[2, 2, 2, 2]
\end{lstlisting}

\end{document}