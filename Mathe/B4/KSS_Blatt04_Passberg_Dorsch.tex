 \documentclass[a4paper]{scrartcl}

\usepackage[utf8]{inputenc}
\usepackage[ngerman]{babel}

\usepackage{url,amssymb,mathrsfs,enumerate,dsfont}
\usepackage[space,extendedchars]{grffile}
\usepackage{verbatim}
\usepackage{listings}
\usepackage{geometry}
\usepackage{tikz}
\usepackage{etoolbox}
\usetikzlibrary{automata,arrows}
\usepackage{subfigure}
\usepackage[ngerman]{babel}
\usepackage{hyperref}
\usepackage{blindtext}
\usepackage{framed}
\usepackage{paralist}
\usepackage{multirow} 
\usepackage{amsmath}
\usepackage{algorithm}
\usepackage[noend]{algpseudocode}

\def\ojoin{\setbox0=\hbox{$\bowtie$}%
  \rule[-.02ex]{.25em}{.4pt}\llap{\rule[\ht0]{.25em}{.4pt}}}
\def\leftouterjoin{\mathbin{\ojoin\mkern-5.8mu\bowtie}}
\def\rightouterjoin{\mathbin{\bowtie\mkern-5.8mu\ojoin}}
\def\fullouterjoin{\mathbin{\ojoin\mkern-5.8mu\bowtie\mkern-5.8mu\ojoin}}

\usetikzlibrary{arrows,shapes, automata}
\setkomafont{disposition}{\normalfont\bfseries}
\setlength\parindent{0pt}

\title{Mathematik für Informatiker \\ Kombinatorik, Stochastik und Statistik}
\subtitle{Übungsblatt 4}
\author{Tom Paßberg , Iain Dorsch}
\date{}
\begin{document}

\maketitle

\newpage

\section*{Aufgabe 1}
\subsection*{a)}
Der Professor versucht die Menge der Partitionen einer 5 elementigen Menge in 3 Teilen aufzuschreiben.
Er hat bereits die Menge der Partitionen einer 4 elementigen Menge in 3 Teilen notiert.

\subsection*{c)}
Es gilt $ S(3,1) = S(2,1) = S(2,2) = S(3,3) = 1$.
\begin{align*}
    S(3,2) &= S(2,1) + 2 \cdot S(2,2) = 1 + 2 \cdot 1 = 3 \\
    S(4,2) &= S(3,1) + 2 \cdot S(3,2) = 1 + 2 \cdot 3 = 7 \\
    S(4,3) &= S(3,2) + 3 \cdot S(3,3) = 3 + 3 \cdot 1 = 6 \\
    S(5,3) &= S(4,2) + 3 \cdot S(4,3) = 7 + 3 \cdot 6 = 25 \\
\end{align*}

\section*{Aufgabe 3}
\subsection*{a)}
Die Anzahl der Möglichen Endergenbnisse ist $3^3 = 27$. Die Anzahl der möglichen Totalordnungen ist $3! = 6$.
Die Anzahl der Ergebisse bei denen der Spieler verliert ist $3! + 2 = 8$ da der Spieler auch verliert wenn alle Kanten des Dreiecks einen Spielstein der selber Ausrichtung enthalten.
Die Wahrscheinlichkeit das der Spieler verliert ist also $\frac{8}{27} = 29,6\%$. \\
Gewinnwarscheinlichkeit: $\frac{19}{27} = 70,4\%$.
\subsection*{b)}
Halbordnungen auf $\{1,2,3\}$
\begin{align*}
    R_1 &= \{(1,1),(2,2),(3,3)\} \\
    R_2 &= \{(1,1),(2,2),(3,3), (1,2)\} \\
    R_3 &= \{(1,1),(2,2),(3,3), (1,3)\} \\
    R_4 &= \{(1,1),(2,2),(3,3), (2,3)\} \\
    R_5 &= \{(1,1),(2,2),(3,3), (2,1)\} \\
    R_6 &= \{(1,1),(2,2),(3,3), (3,1)\} \\
    R_7 &= \{(1,1),(2,2),(3,3), (3,2)\} \\
    R_8 &= \{(1,1),(2,2),(3,3), (1,2), (2,3), (1,3)\} \\
    R_9 &= \{(1,1),(2,2),(3,3), (3,1), (1,2), (3,2)\} \\
    R_{10} &= \{(1,1),(2,2),(3,3), (1,3), (3,2), (1,2)\} \\
    R_{11} &= \{(1,1),(2,2),(3,3), (2,1), (1,3), (2,3)\} \\
    R_{12} &= \{(1,1),(2,2),(3,3), (2,3), (3,1), (2,1)\} \\
    R_{13} &= \{(1,1),(2,2),(3,3), (3,2), (2,1), (3,1)\} \\
\end{align*}
$R_8$ bis $R_{13}$ sind Totalordnungen.
\section*{Aufgabe 4}
\subsection*{a)}
\begin{align*}
    R = \{(1,1),(2,2),(1,2)\}
    R = \{(1,1),(2,2)\}
\end{align*}

\subsection*{c)}
Die Anzahl der Totalordnungen auf einer $n$ elementigen Menge ist $n!$. Das ergibt sich daraus, dass es $n!$ mögliche 
Reihenfolgen gibt eine $n$ elementige Menge anzuordnen. Eine Totalordnung weißt allen Elementen eine eindeutige Reihenfolgen zu.



\section*{Aufgabe 5}

\textbf{Code:}
\begin{lstlisting}
fn partitions(n: u8, m: u8) -> Vec<Vec<Vec<u8>>> {
    if m == 0 || n == 0 {
        return Vec::new();
    }
    if m == 1 {
        return vec![vec![(1..=n).collect()]];
    }

    let mut result = Vec::new();
    for p in partitions(n - 1, m - 1).iter_mut() {
        p.push(vec![n]);
        result.push(p.clone());
    }
    for p in partitions(n-1, m) {
        for i in 0..p.len() {
            let mut p = p.clone();
            p[i].push(n);
            result.push(p);
        }
    }
    result
}
\end{lstlisting}

\textbf{Funktionsaufruf:}
\begin{lstlisting}
let parts = partitions(5, 3);

for (i, part) in parts.iter().enumerate() {
    println!("{:3}: {:?}", i, part);    
}
\end{lstlisting}

\textbf{Ausgabe:}
\begin{lstlisting}
    0: [[1, 2, 3], [4], [5]]
    1: [[1, 2, 4], [3], [5]]
    2: [[1, 2], [3, 4], [5]]
    3: [[1, 3, 4], [2], [5]]
    4: [[1, 3], [2, 4], [5]]
    5: [[1, 4], [2, 3], [5]]
    6: [[1], [2, 3, 4], [5]]
    7: [[1, 2, 5], [3], [4]]
    8: [[1, 2], [3, 5], [4]]
    9: [[1, 2], [3], [4, 5]]
   10: [[1, 3, 5], [2], [4]]
   11: [[1, 3], [2, 5], [4]]
   12: [[1, 3], [2], [4, 5]]
   13: [[1, 5], [2, 3], [4]]
   14: [[1], [2, 3, 5], [4]]
   15: [[1], [2, 3], [4, 5]]
   16: [[1, 4, 5], [2], [3]]
   17: [[1, 4], [2, 5], [3]]
   18: [[1, 4], [2], [3, 5]]
   19: [[1, 5], [2, 4], [3]]
   20: [[1], [2, 4, 5], [3]]
   21: [[1], [2, 4], [3, 5]]
   22: [[1, 5], [2], [3, 4]]
   23: [[1], [2, 5], [3, 4]]
   24: [[1], [2], [3, 4, 5]]
\end{lstlisting}

\end{document}